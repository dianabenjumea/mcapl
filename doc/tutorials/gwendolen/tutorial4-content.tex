\label{tutorial:gwendolen:debugging}
This is the fourth in a series of tutorials on the use of the
\gwendolen\ programming language.  This tutorial looks at some of the
things that typically cause errors in \gwendolen\ programs and how to
identify and fix the
errors.\index{Gwendolen}\index{Gwendolen!debugging}\index{debugging}\index{debugging!Gwendolen} 

For this tutorial we will be working with files from previous
tutorials but editing them to introduce errors.  You may wish to
create a separate folder, \texttt{tutorial4} for this work and copy
files into it.  Remember to update the paths in your configuration
files if you do so. 

\gwendolen\ does not have its own debugger, however you can get a long
way using error outputs and logging information.\index{logging} 

\section{Path Errors}\index{debugging!could not find file}\index{Gwendolen}
If you supply the wrong path or filename in a
configuration\index{Gwendolen!configuration}\index{configuration!gwendolen}
file \gwendolen\ will not be able to find the program you want to run.
You will see an error similar to the following:  
\begin{verbatim}
ail.mas.AIL[SEVERE|main|3:24:57]: Could not find file.  Checked: 
 /src/examples/gwendolen/tutorials/tutorial3/pickuprubble_achiev.gwen,
 /Users/lad/src/examples/gwendolen/tutorials/tutorial3/pickuprubble_achiev.gwen,
 /Users/lad/Eclipse/mcapl/src/examples/gwendolen/tutorials/tutorial3/pickuprubble_achiev.gwen 
\end{verbatim}

\gwendolen\ looks for program files on\index{Gwendolen}
\begin{enumerate}
\item The absolute path given in the configuration file --
  \index{configuration!gwendolen}\index{Gwendolen!configuration} 

\texttt{/src/examples/gwendolen/tutorials/tutorial3/pickuprubble\_achiev.gwen}
above. 
\item The path from the \texttt{HOME} environment variable (normally the user's home directory on Unix systems) -- 

\texttt{/Users/lad/src/examples/gwendolen/tutorials/tutorial3/pickuprubble\_achiev.gwen} above.
\item The path from the directory from which the \java\ program
  \texttt{ail.mas.AIL} is called  -- \index{AIL (class)} 

\texttt{/Users/lad/Eclipse/mcapl/src/examples/gwendolen/tutorials/tutorial3/pickuprubble\_achiev.gwen}
above. 
\item and the path from \texttt{AJPF\_HOME}\index{AJPF\_HOME} if that
  environment variable has been set -- not shown above. 
\end{enumerate}

These should provide sufficient information to appropriately correct
the path name. 

\section{Parsing Errors}\index{Gwendolen}\index{Gwendolen!parsing error}

Parsing errors typically arise because of failures to punctuate your
program correctly.  E.g., failing to close brackets, missing out
commas or semi-colons etc. 

This is the output that arises if you remove the comma between
\lstinline{possible_rubble(X, Y)} and \lstinline{~no_rubble(X, Y)} in
\texttt{pickuprubble\_achieve.gwen} from Tutorial 3.\index{mismatched
  input} 

\begin{verbatim}
line 36:47 mismatched input '~' expecting SEMI
\end{verbatim}
followed by the program nevertheless attempting to execute and failing to terminate, so you will need to kill this.

The first line of this output is from the parser.  This identifies the
line number (36) and character in the line (47) where the error was
first noticed.  It highlights the character that has caused the
problem, \texttt{\~} and then makes a guess at what it should have
been.  In this case the guess is incorrect.  It suggests a semi-colon,
\texttt{SEMI}, when a comma is needed.  This first line is frequently
the most useful piece of output for parsing
errors so it is worthwhile watching for these kinds of errors at the start when you try to execute a program.\index{Gwendolen}\index{mismatched input} 

Parsing errors can also cause plans to fail to apply.  For instance, if we delete the comma from between
\texttt{X} and \texttt{Y} in the guard of the first plan of this
program we get the following output:\index{Gwendolen} 

\begin{verbatim}
line 44:51 extraneous input 'Y' expecting CLOSE
ail.semantics.operationalrules.GenerateApplicablePlansEmptyProblemGoal[WARNING|main|3:19:44]: 
Warning no applicable plan for goal _aholding(rubble)() 
\end{verbatim}

In this case we once again get some useful output from the parser
giving us the line and position in the line where the error occurred.
It is followed by a warning that no plan can be found to match the
goal \texttt{holding(rubble)}.  That is because this is the plan that
didn't parse.\index{no applicable plan} 

\subsection{Some
  Exercises}\index{Gwendolen!debugging!exercises}\index{debugging!Gwendolen} 
Experiment with adding and deleting syntax from your existing
programming files and get used to the kinds of parsing errors that
they generate.  Remember that where a \texttt{no applicable plan}
warning is generated you will often need to manually stop the program
execution.\index{no applicable plan} 

\section{Why isn't my plan applicable?}\index{no applicable plan}\index{Gwendolen}

As mentioned \intutorial{\gwendolen}{3}{tutorial:gwendolen:guards}
sometimes a plan can fail because the agent has not had time to
process incoming beliefs and perceptions.  We can get more information
about the agent's operation by using log
messages.\index{logging}\index{Gwendolen!logging} 

Add the lines
\begin{verbatim}
log.fine = ail.semantics.AILAgent
\end{verbatim}
to the configuration file for the sample solution to the second exercise \exercise{6}{\gwendolen}{3}{ex:tutorial3}.  This is in
\begin{quote}
\texttt{/src/examples/gwendolen/tutorials/tutorial3/answers/pickuprubble\_ex5.2.ail}
\end{quote}
This generates a lot of output.  If you are using Eclipse you may need
to set Console output to unlimited in Eclipse $\rightarrow$
Preferences $\rightarrow$ Run/Debug $\rightarrow$
Console.\index{Gwendolen}\index{example!pickuprubble}\index{AILAgent
  (class)} 

The output should start
\begin{verbatim}
ail.semantics.AILAgent[FINE|main|4:03:27]: Applying Perceive 
ail.semantics.AILAgent[FINE|main|4:03:27]: robot
=============
After Stage StageE :
[square/2-square(1,1), square(1,2), square(1,3), square(1,4), square(1,5), 
square(2,1), square(2,2), square(2,3), square(2,4), square(2,5), 
square(3,1), square(3,2), square(3,3), square(3,4), square(3,5), 
square(4,1), square(4,2), square(4,3), square(4,4), square(4,5), 
square(5,1), square(5,2), square(5,3), square(5,4), square(5,5), ]
[]
[]
source(self):: 
   *  start||True||+!_aall_squares_checked()()||[]
[] 
\end{verbatim}
This tells us that the agent is applying the rule from the agent's
\emph{reasoning cycle}\index{reasoning cycle} called
\texttt{Percieve}\index{Percieve (rule)} (we will discuss the
reasoning cycle in a later tutorial). 

Then we get the current state of the agent.  It is called,
\texttt{robot}, and we have a list of its beliefs\index{belief} (lots
of beliefs about squares), then a list of goals\index{goal} (none at
the start because it hasn't yet added the initial
goal\index{goal!initial}) and then a list of sent
messages\index{communication}\index{message}\index{message!sent} (also
empty) and then the intentions\index{intention}.  In this case the
initial intention is the \texttt{start}
intention\index{intention!start} and the intention is to acquire the
goal \texttt{all\_squares\_checked} which is an achievement goal (the
\texttt{\_a} at the start of the goal name) -- again we will cover
intentions in a later tutorial.\index{Gwendolen} 

A little further on in the output the agent adds this as a goal:
\begin{verbatim}
ail.semantics.AILAgent[FINE|main|4:10:45]: Applying Handle Add Achieve Test Goal with Event 
ail.semantics.AILAgent[FINE|main|4:10:45]: robot
=============
After Stage StageD :
[square/2-square(1,1), square(1,2), square(1,3), square(1,4), square(1,5), square(2,1), square(2,2), square(2,3), square(2,4), square(2,5), square(3,1), square(3,2), square(3,3), square(3,4), square(3,5), square(4,1), square(4,2), square(4,3), square(4,4), square(4,5), square(5,1), square(5,2), square(5,3), square(5,4), square(5,5), ]
[all_squares_checked/0-[_aall_squares_checked()]]
[]
source(self):: 
   *  +!_aall_squares_checked()||True||npy()||[]
   *  start||True||+!_aall_squares_checked()()||[]
[] 
\end{verbatim}
So you can see that \texttt{all\_squares\_checked} now appears as a goal in the goal list. \index{Gwendolen}

\begin{sloppypar}
If you remove the action \texttt{do\_nothing} from the first plan\ in
\texttt{pickuprubble\_ex5.2.gwen}\index{example!pickuprubble} then you
end up with repeating output of the form: 
\end{sloppypar}

\begin{verbatim}
ail.semantics.AILAgent[FINE|main|4:20:16]: Applying Generate Applicable Plans Empty with Problem Goal 
ail.semantics.AILAgent[FINE|main|4:20:16]: robot
=============
After Stage StageB :
[at/2-at(5,5), , 
checked/2-checked(1,1), checked(1,2), checked(1,3), checked(1,4), checked(1,5), checked(2,1), checked(2,2), checked(2,3), checked(2,4), checked(2,5), checked(3,1), checked(3,2), checked(3,3), checked(3,4), checked(3,5), checked(4,1), checked(4,2), checked(4,3), checked(4,4), checked(4,5), checked(5,1), checked(5,2), checked(5,3), checked(5,4), checked(5,5), , 
holding/1-holding(rubble), , 
rubble/2-rubble(2,2), , 
square/2-square(1,1), square(1,2), square(1,3), square(1,4), square(1,5), square(2,1), square(2,2), square(2,3), square(2,4), square(2,5), square(3,1), square(3,2), square(3,3), square(3,4), square(3,5), square(4,1), square(4,2), square(4,3), square(4,4), square(4,5), square(5,1), square(5,2), square(5,3), square(5,4), square(5,5), ]
[all_squares_checked/0-[_aall_squares_checked()]]
[]
source(self):: 
   *  x!_aall_squares_checked()||True||npy()||[]
   *  +!_aall_squares_checked()||True||npy()||[]
   *  start||True||+!_aall_squares_checked()()||[]
\end{verbatim}

\begin{sloppypar}
Where \texttt{x!\_aall\_squares\_checked()||True||npy()||[]} indicates
that there is some problem with the goal\index{goal!problem} and the
agent is seeking to handle this.  Finding where this problem first
occurred in all the output is something of a chore though it can
sometimes be possible to search forwards through the output for the
first occurrence of \texttt{x!\_}  or for the Warning message.  Here
we see the agent is in the following state:\index{Gwendolen} 
\end{sloppypar}

\begin{verbatim}
ail.semantics.operationalrules.GenerateApplicablePlansEmptyProblemGoal[WARNING|main|4:20:15]: Warning no applicable plan for goal _aall_squares_checked() 
ail.semantics.AILAgent[FINE|main|4:20:15]: Applying Generate Applicable Plans Empty with Problem Goal 
ail.semantics.AILAgent[FINE|main|4:20:15]: robot
=============
After Stage StageB :
[at/2-at(5,5), , 
checked/2-checked(1,1), checked(1,2), checked(1,3), checked(1,4), checked(1,5), checked(2,1), checked(2,2), checked(2,3), checked(2,4), checked(2,5), checked(3,1), checked(3,2), checked(3,3), checked(3,4), checked(3,5), checked(4,1), checked(4,2), checked(4,3), checked(4,4), checked(4,5), checked(5,1), checked(5,2), checked(5,3), checked(5,4), checked(5,5), , 
rubble/2-rubble(2,2), rubble(5,5), , 
square/2-square(1,1), square(1,2), square(1,3), square(1,4), square(1,5), square(2,1), square(2,2), square(2,3), square(2,4), square(2,5), square(3,1), square(3,2), square(3,3), square(3,4), square(3,5), square(4,1), square(4,2), square(4,3), square(4,4), square(4,5), square(5,1), square(5,2), square(5,3), square(5,4), square(5,5), ]
[all_squares_checked/0-[_aall_squares_checked()]]
[]
source(self):: 
   *  +!_aall_squares_checked()||True||npy()||[]
   *  start||True||+!_aall_squares_checked()()||[]

[source(self):: 
   *  +checked(5,5)||True||npy()||[]
, source(percept):: 
   *  start||True||-rubble(5,5)()||[]
, source(percept):: 
   *  start||True||+holding(rubble)()||[]
] 
\end{verbatim}
Here we see the agent believes it is at square (5, 5).  It believes it
has checked all the squares.  But it does not yet believe it is
holding any rubble.  It \emph{does} have an intention to hold rubble: 
\begin{verbatim}
source(percept):: 
    *  start||True||+holding(rubble)()||[]
\end{verbatim}
but it hasn't processed this yet and so hasn't added
\texttt{holding(rubble)} to its belief base.\index{Gwendolen} 

As you learn more about \gwendolen, the reasoning cycle and intentions
you will be able to get more information from this output.  However at
the moment it is important to note it can be useful for seeing exactly
what is in the agent's belief base and goal base at any
time.\index{Gwendolen} 

\subsection{Some Exercises}\index{Gwendolen!debugging!exercises}\index{debugging!Gwendolen}
Run some of your other programs with
\texttt{ail.semantics.AILAgent}\index{AILAgent (class)} set at log
level \texttt{fine} and see if you can get a feel for how an agents
beliefs and goals change as the program executes. 

Note:  If you add 
\begin{verbatim}
pretty = gwendolen
\end{verbatim}\index{Pretty Printer}
you will get a slightly different presentation of the output in a more natural language format.  You may want to experiment with which style of output. you prefer.

\section{Tracing the execution of reasoning rules}\index{logging}\index{reasoning rules}\index{reasoning rules!debugging}\index{Gwendolen}
Another logger that can be useful is the one that traces the
application of \prolog\ reasoning rules.  This can be also be useful
for working out why a plan that \emph{should} apply does not.  Try
adding the line\index{EvaluationAndRuleBaseIterator (class)} 
\begin{verbatim}
log.fine = ail.syntax.EvaluationAndRuleBaseIterator
\end{verbatim}
to \texttt{pickuprubble\_achieve.ail}
\intutorial{\gwendolen}{3}{tutorial:gwendolen:guards}.  If you now run
the program you will get a lot of information about the unification of
the reasoning rule starting with:\index{example!pickuprubble} 

\begin{small}
\begin{verbatim}
ail...[FINE|...]: Checking unification of holding(rubble)() with unifier [] 
ail...[FINE|...]: Checking unification of square_to_check(X,Y)() with unifier [] 
ail...[FINE|...]: Looking for a rule match for square_to_check(X0,Y0) :- (possible_rubble(X0,Y0) & not (no_rubble(X0,Y0))) and square_to_check(X,Y)() 
ail...[FINE|...]: Checking unification of square_to_check(X,Y)() with unifier [] 
ail...[FINE|...]: Checking unification of possible_rubble(X0,Y0) with unifier [X-_VC1, X0-_VC1, Y-_VC2, Y0-_VC2] 
ail...[FINE|...]: Checking unification of possible_rubble(X0,Y0) and <possible_rubble(1,1), > 
ail...[FINE|...]: Unifier for possible_rubble(X0,Y0) and <possible_rubble(1,1), > is [X-1, X0-1, Y-1, Y0-1] 
ail...[FINE|...]: Checking unification of no_rubble(X0,Y0) with unifier [X-1, X0-1, Y-1, Y0-1] 
ail...[FINE|...]: square_to_check(X,Y)() matches the head of a rule. 
ail...[FINE|...]: Rule instantiated with [X-1, X0-1, Y-1, Y0-1] 
ail...[FINE|...]: Checking unification of square_to_check(X,Y)() with unifier [] 
ail...[FINE|...]: Looking for a rule match for square_to_check(X0,Y0) :- (possible_rubble(X0,Y0) & not (no_rubble(X0,Y0))) and square_to_check(X,Y)() 
ail...[FINE|...]: Checking unification of square_to_check(X,Y)() with unifier [] 
ail...[FINE|...]: Checking unification of possible_rubble(X0,Y0) with unifier [X-_VC3, X0-_VC3, Y-_VC4, Y0-_VC4] 
ail...[FINE|...]: Checking unification of possible_rubble(X0,Y0) and <possible_rubble(1,1), > 
ail...[FINE|...]: Unifier for possible_rubble(X0,Y0) and <possible_rubble(1,1), > is [X-1, X0-1, Y-1, Y0-1] 
ail...[FINE|...]: Checking unification of no_rubble(X0,Y0) with unifier [X-1, X0-1, Y-1, Y0-1] 
ail...[FINE|...]: square_to_check(X,Y)() matches the head of a rule. 
ail...[FINE|...]: Rule instantiated with [X-1, X0-1, Y-1, Y0-1] 
ail.mas.DefaultEnvironment[INFO|main|4:44:40]: robot done move_to(1,1) 
\end{verbatim}
\end{small}

This is the selection process for the first plan in the program.  We
discuss it line by line. 
\begin{itemize}
\item First it unifies\index{unification} with the achieve goal
  \texttt{holding(rubble)}.  This does not instantiate any variables
  so there is an empty unifier\index{unifier}, \texttt{[]}.   
\item Then it checks the plan guard\index{plan}\index{plan!guard}
  which is \lstinline{B square_to_check(X, Y)}.   
\item Since there is nothing in the belief\index{belief} base about
  this but there is a reasoning rule\index{reasoning rules} it now
  looks for a  unifier\index{unifier} between these.  Notice how it
  has renamed the variables\index{Gwendolen!variables} in the rule to
  \texttt{X0} and \texttt{Y0} -- this is to avoid errors arising from
  false unifications.   
\item It then attempts to unify \lstinline{B square_to_check(X, Y)}
  with the head of this rule.\index{reasoning rules}\index{Gwendolen} 
\item As a result of this unification\index{unification} \texttt{X}
  and \texttt{X0} are unified and \texttt{Y} and \texttt{Y0} are
  unified.  For technical reasons these are unified via \emph{variable
    clusters}\index{variable cluster}, \texttt{VC1} and \texttt{VC2}
  respectively. 
\begin{sloppypar}
\item It then checks the body of the rule\index{reasoning rules}
  starting with looking for something to unify
  \texttt{possible\_rubble(X0, Y0)}.  This unifies\index{unification}
  with the belief\index{belief} \texttt{possible\_rubble(1, 1)}. 
\end{sloppypar}
\item The unifier\index{unifier} is reported showing all the variables
  are now unified with the number 1. 
\item The system then checks to see if \texttt{no\_rubble(X0, Y0)}
  unifies with anything using this unifier.  The rule\index{reasoning
    rules} will fail if it does match because this predicate was
  negated. 
\item It doesn't match so the rule has matched.\index{Gwendolen}
\item With everything unified to 1.
\item The process then repeats because of the way the reasoning rule
  processes transitions.\index{reasoning rules} 
\end{itemize}

If we look later in the trace we can see the same process being run
after \texttt{no\_rubble(1, 1)} has been added to the
belief\index{belief} base.\index{Gwendolen} 

\begin{verbatim}
ail...[FINE|...]: Checking unification of square_to_check(X,Y)() with unifier [] 
ail...[FINE|...]: Looking for a rule match for square_to_check(X0,Y0) :- (possible_rubble(X0,Y0) & not (no_rubble(X0,Y0))) and square_to_check(X,Y)() 
ail...[FINE|...]: Checking unification of square_to_check(X,Y)() with unifier [] 
ail...[FINE|...]: Checking unification of possible_rubble(X0,Y0) with unifier [X-_VC5, X0-_VC5, Y-_VC6, Y0-_VC6] 
ail...[FINE|...]: Checking unification of possible_rubble(X0,Y0) and <possible_rubble(1,1), > 
ail...[FINE|...]: Unifier for possible_rubble(X0,Y0) and <possible_rubble(1,1), > is [X-1, X0-1, Y-1, Y0-1] 
ail...[FINE|...]: Checking unification of no_rubble(X0,Y0) with unifier [X-1, X0-1, Y-1, Y0-1] 
ail...[FINE|...]: Checking unification of no_rubble(X0,Y0) and <no_rubble(1,1), > 
ail...[FINE|...]: Unifier for no_rubble(X0,Y0) and <no_rubble(1,1), > is [X-1, X0-1, Y-1, Y0-1] 
ail...[FINE|...]: Checking unification of possible_rubble(X0,Y0) with unifier [X-_VC5, X0-_VC5, Y-_VC6, Y0-_VC6] 
ail...[FINE|...]: Checking unification of possible_rubble(X0,Y0) and <possible_rubble(3,3), > 
ail...[FINE|...]: Unifier for possible_rubble(X0,Y0) and <possible_rubble(3,3), > is [X-3, X0-3, Y-3, Y0-3] 
ail...[FINE|...]: Checking unification of no_rubble(X0,Y0) with unifier [X-3, X0-3, Y-3, Y0-3] 
ail...[FINE|...]: Checking unification of no_rubble(X0,Y0) and <no_rubble(1,1), > 
ail...[FINE|...]: square_to_check(X,Y)() matches the head of a rule. 
ail...[FINE|...]: Rule instantiated with [X-3, X0-3, Y-3, Y0-3] 
\end{verbatim}
\begin{sloppypar}
Here after a unifier\index{unifier} is found for \texttt{no\_rubble(1,
  1)} that unifier for the rule has failed and the process backtracks
to look for a different unifier for \texttt{possible\_rubble(X0, Y0)}
in this instance finding (3, 3) and this time the rule succeeds. 
\end{sloppypar}

\section{Conclusion}
Hopefully this tutorial has given you some basic tools for tracking
errors in your \gwendolen\ programs.  Although the logging facilities
generate a lot of output that can be tiresome to read through, they
are occasionally very useful for working out what is going wrong in a
program.  We will look at more debugging possibilities after we have
covered the \gwendolen\ reasoning cycle in a
tutorial.\index{Gwendolen}\index{logging}\index{debugging}\index{Gwendolen!debugging}\index{debugging!Gwendolen} 

